This chapter attempts to capture the reasons for change, and the tensions that is already causing. Malone, an ex central banking analyst now working in crypto, links across the last two chapters of blockchain, and money, \href{https://twitter.com/brendanpmalone/status/1628067806984937472}{in a Twitter thread}. He believes that policymakers should focus on the underlying problems in the financial system, rather than just focusing on crypto. He has a lot of appreciation for US policymakers worrying about risk in the financial system. Crypto gets  attention because it's an easy target, but Malone believes that the real problems are so much bigger. According to Malone, people want to hold USD money to store value and make payments. Most are familiar with cash and bank deposits, but there's actually a spectrum of assets of varying quality that act like money, as we saw in the previous chapter. These include Euro dollars, repo, commercial paper, and more. This is what people are talking about when they reference the shadow banking system - money moving around the financial system outside of traditional banks, primarily in non-banks. Malone notes that some amount of shadow banking activity is good because it allows the money supply to be more reactive and expand and contract with economic activity, which helps fuel economic growth. However, the regulatory and political apparatus and the underlying systems weren't really designed for a system this large, opaque, and multi-dimensional. This was seen in 2008 and 2009, which was as much about shadow banking and financial plumbing as it was about subprime housing and complex derivatives. The same was seen in 2020 with COVID-19.\par
In times of crisis, people want to be able to freely convert whatever they are holding into something safer on the spectrum. Sadly, sometimes market liquidity isn't there, so the central banks and come to save the day, and this kicks the can down the road. The core issue is that people an institutions want to store capital in places they can't access due to technical, institutional, or geopolitical reasons. Sovereigns hold US treasuries, hedge funds and HRTs use repo, and we have seen that the crypto and Bitcoin economies have stablecoins.\par
Since 2008 and 2009, the Treasury Market has gotten significantly larger, more fragile, and more complex. Banks have even more restrictions on creating deposits, and the demand for safe assets keeps skyrocketing. On top of that, the geopolitical landscape has changed dramatically, with US sanctions and seizure of Russian USD assets. Malone notes that crypto is a response to these underlying problems. Although it is not perfect, it is getting better as people learn from past experiences and begin to build regulatory clarity. This issue of regulatory clarity leads us into this section of the book, which looks are implicit or explicit corruption of governance.\
As a uueful example; The New York Magazine article provided an in-depth interview with Gary Gensler, the head of the Securities and Exchange Commission, in which he shared his thoughts on the cryptocurrency industry. One of the key takeaways was his belief that all cryptocurrencies, except for Bitcoin, should be considered securities, as they involve relying on the work of others to give them value. Gensler is an ex banker, and an ambitious politician, with his eyes on bigger prizes. He openly courted the attention of the now disgraced top team at FTX which failed so spectacularly. His assertions have sparked controversy, as it raises questions about the feasibility of registering all tokens as securities, given the unique challenges posed by open-source protocols and the changing nature of blockchain technologies. Critics argue that Gensler's stance could harm innovation and capital formation, as companies and entrepreneurs may struggle to comply with onerous regulations or abandon their projects altogether. The current system simply doesn't fit this new self forming marketplace, and his implication seems to be that the legal end game here is the destruction of the invested capital, because of non compliance. This has led to frustration and concern among crypto advocates and investors, who worry about the impact of such policies on the industry's growth and development.\par
The discourse should be on the much more fundamental questions of the monetary system and fragility of past assumptions and their ability to predict what comes next. Even as these conversations happen, however, the Bitcoin and stable coin builders will keep building because they are not going to sit around and wait for solutions to be presented to them.
\section{Global politics \& digital society}
\subsection{The World Economic Forum}
The World Economic Forum (WEF) is a non-governmental organization founded in 1971 by Klaus Schwab. It is well known for its annual meeting in Davos, Switzerland, where world leaders, CEOs, and various stakeholders gather to discuss global issues and potential solutions. Although the WEF does not have direct control over policymaking, its influence on global policy arises from its role as a platform for dialogue and idea exchange, as well as its ability to bring together influential individuals.\par
As unelected technocrats, the WEF's impact on global policy can be observed through these aspects:
\begin{itemize}
\item Convening power: The WEF's Davos meeting is a high-profile event that attracts prominent political figures, business executives, and other influential individuals. This ability to assemble people allows the WEF to initiate conversations on global issues, create networks, and establish connections among key players. These interactions can lead to ideas and initiatives that might eventually shape global policy.
\item Knowledge sharing and thought leadership: The WEF produces a range of publications, reports, and research that provide insights into various global challenges. By disseminating this knowledge, the WEF contributes to the broader understanding of complex issues and helps to inform policymaking by governments, businesses, and other organizations.
\item Agenda-setting: Through its conferences and publications, the WEF identifies and highlights emerging trends, risks, and opportunities, which can help to set the agenda for global policy discussions. By bringing attention to specific issues, the WEF can indirectly influence the priorities of governments and other decision-makers.
\item Public-private cooperation: The WEF actively promotes collaboration between the public and private sectors in addressing global challenges. By fostering partnerships and facilitating dialogue between these sectors, the WEF can help drive the development and implementation of policies that require cooperation between governments, businesses, and civil society.
\end{itemize}
Despite its influence, critics argue that the WEF's position as unelected technocrats raises concerns about the organization's legitimacy and accountability. They contend that the WEF's ability to shape global policy without being directly answerable to citizens can undermine democratic processes and result in policies that prioritize the interests of elites over the broader public. However, others argue that the WEF's role in facilitating dialogue and collaboration is essential for tackling complex global challenges that require coordinated action across sectors and borders.\par
Interesting for us the WEF recently released its annual \href{https://www3.weforum.org/docs/WEF_The_Global_Risks_Report_2022.pdf}{Global Risks Report}, which highlights various threats and challenges facing the world today, and which intersect with all of the narratives in this book. The report discusses issues related to cybersecurity, public trust, and social cohesion, and underscores the importance of a comprehensive approach to addressing these challenges.\par
The WEF's founder, Klaus Schwab, has previously argued for a ``great reset'' in society and the economy, which involves revamping various aspects of our lives, from education to social contracts and working conditions. This reset would require the construction of new foundations for economic and social systems.\par 
The WEF Global Risks Report 2022 focuses on five main categories, which are also part of their ``Great Narrative for Humankind'' initiative:
\begin{itemize}
\item Economy
\item Environment
\item Geopolitics
\item Society
\item Technology
\end{itemize}
The report emphasizes that the erosion of social cohesion has been a significant global issue since the start of the COVID-19 crisis. In addressing these challenges, the WEF suggests that public-private collaborations are necessary to ensure effective decision-making and to safeguard the future of humanity.\par
The report also highlights the increasing digital dependency that intensifies cyberthreats, as the WEF has long warned of the potential for a significant cyber pandemic. The rapid spread of a cyber attack with ``COVID-like characteristics'' could potentially cause more damage than any biological virus.\par
The WEF Global Risks Report 2022 delves further into the potential consequences of a cyber pandemic. In a section titled ``Shocks to Reflect Upon'' the report explores the possibility of a wide-ranging and costly attack that could lead to cascading failures in systemically important businesses and disrupt services, ultimately undermining digital transformation efforts made in recent year.\par
The report also emphasizes the need for governments to address cyberthreats and warns that without mitigation, the escalation of cyberwarfare and the disruption of societies could result in a loss of trust in governments' ability to act as digital stewards.\par
To better understand the risks associated with technology, the WEF report explores the concept of the fourth industrial revolution, which Schwab believes will lead to the fusion of our physical, biological, and digital identities. This fusion will be facilitated by technologies such as artificial intelligence, internet of things-enabled devices, edge computing, blockchain, and 5G. You can see they're examining similar things to this book.\par
As explaining in this work, these technologies present numerous opportunities for businesses and societies, they also expose users to elevated and more pernicious forms of digital and cyber risk. The report also discusses the potential emergence of the metaverse, which could create new vulnerabilities for malicious actors by increasing the number of entry points for malware and data breaches, again a central theme of this text.\par
In light of these risks, the WEF report suggests that users will need to navigate security vulnerabilities inherent in complex technologies characterized by decentralization and a lack of structured guardrails or sophisticated onboarding infrastructure.\par
The report also touches on the issue of digital identity as we do. They view digital identity is a crucial component of accessing products, services, and information in a digital world, but again, this raises concerns about privacy, security, and the potential for misuse.\par
Finally, the WEF Global Risks Report 2022 addresses the issue of public trust, noting that the growth of deepfakes and disinformation-for-hire can deepen mistrust between societies, businesses, and governments. To rebuild trust and social cohesion, the report calls for leaders to adopt new models, look long term, renew cooperation, and act systemically.\par 
It's absolutely crucial to note that the WEF is a powerful organisation, with global sway over policy, and is an enormous concentration of power in the hands of unelected technocrats. The authors are very sceptical of the WEF, but this report highlights what both technocrats and policy makers are thinking.
\subsection{Money and The State}
It seems a pretty reasonable that the best `systemic' approach is a separation between major centralising forces such as state, church, and money. In practice we can see that globally, this isn't the case, with bad hotspots of high corruption where all three meld together into kleptocratic dictatorships, or theocracies. For our purposes in the UK it's useful to look at the concept of `austerity'.\par
Austerity is a term used to describe a set of economic policies that aim to reduce government spending and debt, often through cuts to public services and welfare programs. The concept of austerity has its origins in the 1920s, following the end of World War I and the economic crisis that ensued. In the wake of the war, many Western European countries were struggling with high levels of debt and inflation. In response, governments began implementing policies to reduce spending and balance their budgets.\par
We have seen in the previous chapter that the concept of inflation itself is complex, and somewhat argued about still. In the 1920s, Keynes was one of the first to argue against austerity measures, arguing that cutting government spending during a recession would only worsen the economic downturn. Instead, he advocated for increased government spending to stimulate economic growth and reduce unemployment. Despite this, many governments continued to implement austerity policies throughout the 1920s and 1930s.\par
In the post-World War II period, the rise of the welfare state and the adoption of Keynesian economic policies led to a shift away from austerity in many countries. However, in the 1970s, a new economic crisis led to a resurgence of austerity policies, particularly in the United States and United Kingdom. In the 1980s, the rise of neoliberalism and the influence of economists such as Milton Friedman led to further cuts to government spending and the rolling back of the welfare state.\par 
Today, the concept of austerity continues to shape economic policy, particularly in the wake of the 2008 financial crisis. Many governments, particularly in Europe, have implemented austerity measures in response to the crisis, leading to cuts to public services and welfare programs. The effectiveness of these policies remains a contentious issue, with some arguing that they have helped to reduce debt and stabilize economies, while others argue that they have led to increased inequality and hindered economic growth. Looking around at the state of the world, and the widening gap between the rich and the poor, it is possible to have some sympathy with those who see patterns in the bahaviour of political leaders and the controllers of Western capital and global resources. The system seems engineered to reward a few. It is possible to view `austerity' as a means of political control of economic levers, in order to de-democratise populations. This mantra of `do more, consume less' has perhaps become a defacto methodology to constrain popular ideas, diverting capital back into the hands of incumbents, land owners, and the politically and economically motivated \cite{mattei2022capital}. It seems that the controlling nexus of this political framework globally is the concept of the central bank, unelected technocrats whose tenures span across political administrations. Again, this can be traced back to the 1920's. Hawtrey’s 1925 ``Currency \& Public Administration'' asserts that a central bank should \textit{``Never explain; never regret; never apologise.''}, and speaks glowingly of the selfish market \cite{hawtrey1925currency}. This economic model is referred to as Dirigisme and feels increasingly the global norm \cite{balassa2013theory}.  We can perhaps here see the divergent point at which the lionization of the market began. Again, to be clear, the authors are not economists, but it does seem that in a global digital society there is room to explore more equitable models of global value, governance, and trust.\par
Remember that these centrally planned national and global actions provide liquidity to the private banking sector. Like the digital money analogues discussed earlier in the book private banks operate fractional reserve banking. This is a banking system where banks hold only a fraction of the deposits they receive as reserves, while the rest is lent out to customers. This means that the money supply in an economy can be increased through the lending activities of banks (itself a complex inflationary force which devalues money over time, feeding back into the policy directives of the central banks. The fractional reserve system is useful for capital creation in times of growth, but relies on the confidence of the depositors. Historical examples of bank runs which threatened systemic risk or caused failures of the banking system include:
\begin{itemize}
\item The Bank of United States crisis in the 1930s: This was the largest bank failure in American history and was a result of a bank run caused by rumors of financial mismanagement.
\item The Savings and Loan crisis of the 1980s: This was a result of a large number of failed savings and loan associations in the United States, which were caused by a combination of factors including poor management, risky lending practices, and a decline in real estate values.
\item The Nordic banking crisis of the 1990s: This crisis was caused by a combination of factors including a real estate bubble, high levels of debt, and a lack of regulation. It resulted in the collapse of several major banks in Sweden, Finland, and Norway, and had a significant impact on the economies of the region.
\item The Bank of Japan crisis in the late 1990s: This crisis was caused by a combination of factors including a real estate bubble, high levels of debt, and a lack of regulation. It resulted in the collapse of several major banks and had a significant impact on the Japanese economy.
\item The Asian Financial Crisis of 1997: This crisis was triggered by a devaluation of the Thai baht and quickly spread throughout the region, causing a number of major banks to fail. The crisis was largely a result of a lack of transparency and poor regulation in the banking industry.
\item The 2008 financial crisis in Iceland: This crisis was caused by the collapse of the country's three largest banks, which had been engaging in risky lending practices and had accumulated large amounts of debt. The crisis had a devastating impact on the Icelandic economy and resulted in a severe recession.
\item The Global Financial Crisis of 2007-2009: This was a result of a widespread failure of the global banking system, caused by a combination of factors including the housing market collapse, risky lending practices, and a lack of regulation.
\item The collapse of Banco Popular in Spain in 2017: This was one of the largest bank failures in European history, and was caused by a combination of factors including a large amount of bad debt and a declining real estate market.
\item There were many bank runs on smaller rural banks in China during 2022. The financial conditions of Chinese banks are somewhat reminiscent of the 2008 American landscape.
\end{itemize}
In response to the Global Financial Crisis, many measures have been taken to shore up the banking system, including the creation of new regulatory bodies, the implementation of new regulations, such as the Dodd-Frank Wall Street Reform and Consumer Protection Act, which increased the regulatory oversight of the banking industry. The introduction of stress testing for banks, to ensure that they have enough capital to withstand financial shocks, globally, has radically deleveraged banks from around 1:40 fractional reserve, to around 1:10.\par 
There is increased political pressure to regulate the banking industry and prevent another financial crisis. However, there is also political opposition to excessive regulation, as some argue that it may stifle economic growth. There are concerns about rising levels of debt and the potential for another financial crisis.\par 
It's interesting that Brett, a former FDIC regulator \href{https://blog.orchid.com/exfdic-regulator-on-trust-and-the-battle-of-the-social-media-videos/}{believes that} the 2008 US bank run was sparked by youtube posts of queues forming at banks. He says those that formed the initial lines carried memories of the great depression, but that once youtube started showing the footage more broadly the contagion struck. In the world of instant messaging media today we can perhaps see how this might happen again. More recently, the 2023 `wobble' in global banking caused by the collapse of America's 5th largest \href{https://theconversation.com/why-svb-and-signature-bank-failed-so-fast-and-the-us-banking-crisis-isnt-over-yet-201737}{bank SVB} has precipitated strong intervention by the federal government, who have opted to `backstop' investor deposits. In the midst of this potential crisis it it notable that TikTok (now arguably the world's \href{https://blog.cloudflare.com/popular-domains-year-in-review-2021/}{most popular search engine}) is carrying millions of hashtag references to \href{https://www.tiktok.com/tag/bankrun?lang=en}{bankruns}. Senator Kelly in the USA \href{https://public.substack.com/p/exclusive-senator-mark-kelly-called}{allegedly inquired} about the potential for limiting such references on social media, and a UK minister is \href{https://news.sky.com/story/tiktok-ban-minister-asks-national-cyber-security-centre-to-look-into-safety-of-app-12833371}{asking for security services} to examine the risks of the Chinese application. The perhaps reflects concern about algorithmically driven geopolitically motivated threats to the banking system.\par
There is a growing awareness of the role of banks in the economy, and a growing desire for greater transparency and accountability. There is also a growing mistrust of banks, particularly in light of the Global Financial Crisis. As we have seen, the advent of new technologies, such as blockchain CBDC, and fintech, is changing the way that banks operate and interact with customers. This presents both opportunities and challenges for the banking industry. As a final controversial aside, there is \href{https://apnews.com/article/signature-bank-fdic-barney-frank-silicon-valley-6ad86262d9945675a42d735b66ace4f2}{industry suspicion} that the collapse of SVB has been used as cover to close the final US bank servicing crypto, effectively decapitating the banking rails of the industry, and forcing it overseas. Were it not for the credibility of the people making these claims, this would seem pretty wild, but the prevailing winds are surely blowing against the disruptive potential of a money system which is beyond the control of legislators.
\subsection{Surveillance Capitalism}
Surveillance capitalism is a term coined by Harvard Business School professor Shoshana Zuboff to describe the business model of using data collected from individuals to target advertising and influence behavior. The concept of surveillance capitalism emerged in the late 20th and early 21st centuries with the rise of technology companies that specialize in gathering and analyzing personal data.\par
The history of surveillance capitalism can be traced back to the early days of the internet. In the 1990s, companies such as DoubleClick and Omniture began collecting data on internet users' browsing habits in order to target advertising. As the internet grew in popularity, these companies were able to gather an increasing amount of data on individuals, allowing them to more effectively target advertising and increase profits.\par
The advent of smart phones and mobile technology in the 2000s further expanded the reach of surveillance capitalism. With the widespread adoption of smart phones and mobile apps, companies were able to collect even more data on individuals, including location data and information about their physical activity. This data was used to target advertising and influence behavior, leading to the rise of companies such as Google and Facebook, which have become dominant players in the digital advertising market.\par
The use of data collected from individuals to influence behaviour has also been used to influence political campaigns. In the 2016 US presidential election, Cambridge Analytica, a data analytics firm, used data collected from Facebook users to influence voter behaviour. The firm used the data to target advertising and create psychological profiles of individuals, allowing them to more effectively influence voter behaviour.\par
The business model of surveillance capitalism has been widely criticized for its ethical implications. Critics argue that the collection and use of personal data without consent is a violation of individuals' privacy and that the use of data to influence behaviour is manipulative and unethical. In recent years, there have been calls for greater regulation of the tech industry to address these concerns.\par 
Surveillance capitalism has led to significant compliance overheads for companies that collect and use personal data. There are a number of laws and regulations that have been put in place to protect individuals' privacy, such as the General Data Protection Regulation (GDPR) in the European Union and the California Consumer Privacy Act (CCPA) in California, USA. These laws require companies to obtain consent from individuals before collecting and using their data, and to provide individuals with the right to access, correct, and delete their data.\par 
Complying with these laws can be costly and time-consuming for companies. They may need to hire additional staff to handle data privacy compliance, and may also need to invest in new technology to manage and protect personal data. In addition, companies are at risk of significant fines if they fail to comply with these laws.\par
In terms of who profits from surveillance capitalism, the primary beneficiaries are technology companies such as Google and Facebook, which have become dominant players in the digital advertising market. These companies collect and analyse large amounts of personal data, which they use to target advertising and influence behavior. This allows them to generate significant profits from advertising revenue.\par
On the other hand, those who suffer the most negative impact from surveillance capitalism are individuals, whose personal data is collected and used without their consent. They are also at risk of their privacy being violated, and their personal data being misused. Additionally, the collection and use of personal data can lead to the manipulation of individuals' behaviour and decision-making, which can have negative consequences for their lives and society at large.\par
Moreover, the business model of surveillance capitalism has also been criticized for creating a power imbalance between companies and individuals. Companies have access to vast amounts of personal data, which they can use to influence behavior and make decisions that affect individuals' lives. This can lead to a lack of privacy and autonomy for individuals, and can also lead to discrimination and bias in decision-making.\par 
This is collectively an erosion of the demarcation between data, state surveillance, banking, and political leadership globally.\par
The term "surveillance state" refers to a state in which government agencies have the power to collect and analyze large amounts of personal data, often without the consent of individuals. The rise of surveillance capitalism has led to concerns about the potential for the creation of a surveillance state, as government agencies may use the data collected by companies for surveillance purposes.\par 
There have been instances where government agencies have used data collected by companies for surveillance purposes. For example, in the United States, the National Security Agency (NSA) has been accused of using data collected by companies such as Google and Facebook for surveillance purposes. The agency's PRISM program, which was revealed by Edward Snowden in 2013, was designed to collect and analyze data from internet companies in order to identify and track individuals. Europe is \href{https://www.patrick-breyer.de/en/posts/chat-control/}{clear about it's intentions} to mandate their complete access to all encrypted personal communications in forthcoming legislation.\par
The use of data collected by companies for surveillance purposes can have significant implications for individuals' privacy and civil liberties. It can also lead to a lack of transparency and accountability, as government agencies may use the data without the knowledge or consent of individuals. In addition, the use of data for surveillance purposes can lead to discrimination and bias in decision-making, as well as a chilling effect on free speech and the exercise of other rights.\par
Akten has been \href{https://memoakten.medium.com/all-watched-over-by-machines-of-loving-grace-8c2464aa6fda}{talking about} the phase transition from digital surveillance to pernicious corporate AI in terms of a modern `religion' for many years \cite{bayer2023artificial}. He feels that despite public awareness of privacy invasion, there has been no significant outcry or unanimous demand for privacy. Instead, most individuals seem to find comfort in the belief that a higher force is watching and protecting the virtuous, while punishing wrongdoers. The concept of a `digital deity' emerge from his thinking in this context, reflecting the role that religion and traditional gods have played in providing ethical frameworks, security, discipline, power, and other societal functions. As societies shift towards materialism and technological dependence, traditional gods lose their relevance, and the need for a new form of overseer arises. This digital deity, existing within the realm of technology and the cloud, perhaps represents an adaptation of primal human belief systems. This well be explored futher in the AI/ML chapter later.\par 
In conclusion, the rise of surveillance capitalism has led to concerns about the potential for the creation of a surveillance state, or worse, a new kind of omnipresent digital culturla authority. Corporations and government agencies may use the data collected by companies for surveillance purposes. This can have significant implications for individuals' privacy and civil liberties. It's important for laws and regulations to be in place to safeguard citizens' rights and privacy in regards to the use of data by government agencies, and to hold them accountable for any misuse of data, and yet it seems the reality of the situation in `post Snowden' seems far from that.\par 
Surveillance Capitalism. As a quick round-up of this area, which is best researched elsewhere:
\begin{itemize}
\item The global digital advertising market is expected to reach \$335 billion by 2023.
\item In 2020, Google and Facebook accounted for 60\% of the global digital advertising market.
\item The data brokerage industry, which includes companies that collect and sell personal data, is estimated to be worth \$200 billion.
\item In 2020, Google and Facebook were reported to have data on over 4 billion active users.
\item As of 2021, the number of data breaches reported worldwide has grown from 4.1 billion in 2018 to 4.9 billion in 2020.
\item In 2013, it was revealed that the US National Security Agency (NSA) had been collecting the phone records of millions of Americans under its PRISM program.
\item In 2013, Edward Snowden leaked classified documents that revealed the scale of the NSA's surveillance programs.
\item In the US, the Foreign Intelligence Surveillance Act (FISA) allows the government to conduct surveillance on non-US citizens outside the US without a warrant.
\item The UK's Investigatory Powers Act 2016, also known as the "snooper's charter," gives government agencies wide-ranging powers to collect and analyze personal data.
\item In 2021, it was reported that the Chinese government has been collecting and analyzing the data of its citizens through a system of "social credit" scores, which are used to monitor and control individuals' behaviour.
\item Surveillance capitalism refers to the business model of collecting and analyzing personal data for the purpose of targeted advertising and other forms of monetization.
\item A recent study by the Center for Digital Democracy found that the top 100 global digital media companies are projected to generate over \$1 trillion in revenue by 2020, much of which is derived from surveillance-based advertising.
\item The number of surveillance cameras in use worldwide is estimated to be over 1 billion, with the majority located in China.
\item A 2018 study by Comparitech found that the average person in the UK is captured on CCTV cameras over 300 times per day.
\item According to a report by the American Civil Liberties Union (ACLU), the FBI has access to over 640 million photographs for facial recognition searches, including driver's license and passport photos.
\item The U.S. government's use of surveillance technologies, such as drones and mass data collection, has been a subject of ongoing controversy and debate.
\item Some experts warn that the increasing use of surveillance technologies by governments and private companies could lead to the erosion of privacy rights and the creation of a "surveillance state."
\item In the USA senate hearing following the collapse of FTX Rep. Jesus Garcia described bitcoin and crypto as an industry that operates outside of the law and relies on hype, implying that the communities that have adopted bitcoin are ill-informed and vulnerable.
\item Bitcoin has been adopted by a variety of communities worldwide, particularly in countries such as Vietnam, the Philippines, Ukraine, India, Pakistan, Brazil, Thailand, Russia, and China.
\item There is an outsized level of adoption among Black Americans in the United States. This trend is not a result of targeted advertising by companies such as FTX, but rather a response to a legacy financial system that has limited individuals' potential.
\item Marginalized early adopters of bitcoin still constitute a minority in their communities, but the worldwide adoption trend among these groups is on the rise.
\item The solutions that outsiders build in bitcoin will ultimately be the source of the technology's promised revolution. Adoption in Africa and possibly India seems likely to be capable of driving this.
\item The paradigm shift will come from those who bring local, real-world focused use cases to their communities, separating bitcoin from the empty hype of speculation.
\item Marginalized communities will lead the industry's recovery and redefine the purpose of bitcoin in the future.
\end{itemize} 

Much of the following text is paraphrased from the work of Guy Turner of `The Coin Bureau', and Lawyer and academic Eden Moglen, and needs more work because of it's critical importance to the book.\par 
The adoption of printing by Europeans in the 15th century led to concerns around access to printed material. The right to read and the right to publish were central subjects in the struggle for freedom of thought for most of the last half millennium. The basic concern was for the right to read in private and to think, speak, and act based on a free and uncensored will. The primary antagonist for freedom of thought at the beginning of this struggle was the universal Catholic Church, an institution aimed at controlling thought in the European world through weekly surveillance of individuals, censorship of all reading material, and the ability to predict and punish unorthodox thought. In early modern Europe, the tools available for thought control were limited, but they were effective. For hundreds of years, the struggle centered around the book as a mass-manufactured article in Western culture, and whether individuals could print, possess, traffic, read, or teach from books without the permission or control of an entity empowered to punish thought. By the end of the 17th century, censorship of written material in Europe began to break down in waves throughout the European world, and the book became an article of subversive commerce, undermining the control of thought.\par
Currently, a new phase in human history is beginning as we are building a single extraneous digital nervous system, that will connect every human mind. Within two generations, every single human being will be connected to this network, in which all thoughts, plans, dreams, and actions will flow as nervous impulses. The fate of freedom of thought and human freedom as a whole will depend upon the organization of this network. Our current generation is the last in which human brains will be formed without contact with this network, and from now on, every human brain will be formed from early life in direct connection to the network, with input from generative AI/ML systems. This possibly results in humanity becoming a super organism of a sort, where each of us is but a neuron in the brain. Unfortunately, this generation has been raised to be consumers of media, which is now consuming us.\par Anonymous reading is being determined against. Efforts discussed throughout the book to ensure privacy, from Zimmerman and the cypherpunks onward, have been met with resistance from government efforts to monitor and control information flow. The outcome of the organization of this network, and the freedom it allows, is currently being decided by this generation.\par
It is not solely the ease of surveillance, nor solely the permanence of data, that is concerning, it is the relentless nature of living after the ``end of forgetting''. Today's encrypted traffic, which is used with relative security, will eventually be decrypted as more data becomes available for crypto analysis. This means that security protocols will need to be constantly updated and redone. Furthermore, no information is ever truly lost, and every piece of information can be retained and eventually linked to other information. This is the rationale behind government officials who argue that a robust social graph of the United States is needed. The primary form of data collection that should be of most concern is media that is used to spy on us, such as books that watch us read them and search boxes that report our searches to unknown parties. There is a lot of discussion about data coming out of Meta/Facebook, but the true threat is code going in. For the past 15 years, enterprise computing has been adding a layer of analytics on top of data warehouses, which is known as business intelligence. This allows for the vast amount of data in a company's possession to be analyzed and used to answer questions the company did not know it had. The real threat of Facebook is the business intelligence layer on top of the Facebook data warehouse, which contains the behaviour of nearly a billion people. Intelligence agencies from around the world want to access this layer in order to find specific classes of people, such as potential agents, sources, and individuals that can be influenced or tortured. The goal is to run code within Facebook to extract this information, instead of obtaining data from Facebook, which would be dead data once extracted. Facebook wants to be a media company and control the web, but the reality is the true value of Facebook is the information and behavior of it's users, and the ability to mine that data.
Distributed internet protocols are important in the context of government overreach into digital society and people's private lives because they provide a level of decentralization and resilience that can help protect against censorship and surveillance.\par
For example, if a government were to attempt to censor or block access to a centralized internet service, it could potentially do so with relative ease. However, if that same service were distributed across a network of nodes, it would be much more difficult for the government to effectively censor or block access to it.\par
Another advantage of distributed protocols is that they are typically more resilient to attacks or failures. If one node in the network goes offline or is compromised, the others can continue to operate, ensuring that the service remains available. This can be especially important in situations where the internet is being used for critical communication, such as during a natural disaster or political crisis.\par
In addition to their benefits for censorship resistance and resilience, distributed protocols can also help protect people's privacy. Because they do not rely on centralized servers or infrastructure, they can be more difficult for governments or other entities to monitor or track. This can be especially important in countries where government surveillance is prevalent or where individuals may be at risk of persecution for their online activities.\par 
There are a number of distributed protocols that have been developed specifically to address issues of censorship and privacy, and these will be covered in more detail later.\par
It is important to note that distributed protocols are not a silver bullet for censorship or privacy concerns. They can be vulnerable to certain types of attacks, such as those that target the nodes of the network, and they may not always be practical for certain types of applications. However, they do provide an important tool for those seeking to protect their freedom of expression and privacy online. They offer a valuable tool for those seeking to protect their freedom of expression and privacy online, and they will likely continue to play a critical role in the future of the internet.\par
In recent years, several countries have proposed or passed bills that would result in unprecedented levels of online censorship. One such example is Canada's Bill C-11, also known as the Online Streaming Act. This bill was first proposed in November 2020 as Bill C-10, but failed to pass due to its controversial provisions. It was reintroduced in February 2021 as Bill C-11 and was approved by the Canadian House of Commons, the first step in the process of becoming law. If passed, the bill would give the Canadian Radio, Television and Telecommunications Commission (CRTC) the power to decide what content Canadians can view on YouTube and other social media platforms. The CRTC would also have the power to dictate what content creators can produce, with a focus on promoting "Canadian content." Additionally, the bill would require certain broadcasters to contribute to the Canada Media Fund, which is used to fund mainstream media in Canada. The bill is currently being considered by the Canadian Senate, which will vote on it in February. If passed, it will then be debated by the Canadian Parliament. Tech companies such as YouTube have reportedly failed to convince the Senate to exclude user-generated content from the bill, indicating a high likelihood of it becoming law. The potential impact on the internet and free expression in Canada is significant, as the bill would give the government significant control over online content and restrict the ability of individuals to share their views and perspectives.\par
The European Union (EU) has separated its online censorship efforts into two separate bills: the Digital Markets Act and the Digital Services Act. These bills were introduced in December 2020 and are part of the EU's Digital Services package, which aims to be completed by 2030. The Digital Services package is the second phase of the EU's digital agenda, which is being enforced through regulation in the public sector and through ESG investing in the private sector. Both the Digital Markets Act and the Digital Services Act were passed in spring 2022 and went into force in autumn 2022, but will not be enforced until later this year and early next year, depending on the size of the relevant entity. The Digital Markets Act aims to increase the EU's competitiveness in the tech space by imposing massive fines on "gatekeepers," or companies that maintain monopolies by giving preference to their own products and services. This could open the door to innovation in cryptocurrency in the EU, but also requires gatekeepers to provide detailed data about the individuals and institutions using their products and services to the EU. The Digital Services Act, on the other hand, aims to regulate the content that is available online, including user-generated content. It does this by requiring companies to remove illegal content within one hour of it being reported and by imposing fines for non-compliance. The act also requires companies to implement measures to protect users from illegal content and from "other forms of harm," which is defined broadly and could include a wide range of content. The EU is also in the process of passing the Artificial Intelligence Regulation Act, which will be discussed later this year and is reportedly the first of its kind. All five bills in the EU's Digital Services package are regulations, meaning they will override the national laws of EU countries. The potential impact on the internet and free expression in the EU is significant, as the Digital Services Act would give the government significant control over online content and restrict the ability of individuals to share their views and perspectives.\par
In the United States, two significant documents related to online censorship are the Kids Online Safety Act and the Supreme Court case Gonzalez v. Google. The Kids Online Safety Act was introduced in February 2021 and is expected to pass later this year due to bipartisan support. The act requires online services to collect Know Your Customer (KYC) information to ensure that they are not showing harmful content to minors. It also gives the Federal Trade Commission (FTC) the power to decide when children have been made unsafe online and allows parents to sue tech companies if their children have been harmed online. The act has received criticism from both sides of the political spectrum and entities outside of Congress, as it is seen as giving too much power to the government to regulate online content and could lead to increased censorship by tech companies.\par
The Supreme Court case Gonzalez v. Google involves the question of whether Google's algorithmic recommendations supported terrorism and contributed to the 2015 terrorist attacks in Paris. The case has been picked up by the Supreme Court after being passed up by various courts of appeal. It is being heard alongside another case, Twitter v. Tumne, involving the role of Twitter's algorithms in a terrorist attack in Istanbul. There are two potential outcomes for the case. If the Supreme Court sides with Gonzalez, it could increase the liability of social media companies under Section 230 of the Communications Decency Act, which allows them to moderate content to a limited extent without violating the First Amendment. Alternatively, the Supreme Court could declare Section 230 unconstitutional, which would make online censorship illegal but also hinder the use of algorithms on the internet. The ideal outcome, in theory, would be for the Supreme Court to side with Google and for Congress to change Section 230. However, giving Congress the power to change the law could lead to increased censorship and the potential for abuse of power.\par
In the UK forthcoming legislation will see tech company leaders liable for \href{https://www.independent.co.uk/news/uk/politics/bill-mps-iain-duncan-smith-molly-russell-rishi-sunak-b2263353.html}{prison sentences} if they fail in their duty to protect minors. This will doubtless lead to both stringent universal requirements for identity proof (KYC), and significantly muted and controlled content on the platforms.\par
Our research focuses on business to business use cases for distributed technologies, and will provide mechanisms for verifying who is communicating with whom, to avoid falling foul of these swinging global infringements on privacy.\par
It is the opinion of this book that information should be free \cite{swartz2008guerilla}
\section{Government over-reach through bureaucracy}
As an contextual example of the soft power which political apparatus uses to influence emergent human behaviour and their markets it is useful to look again to the USA. In 2013, the Obama Administration, faced with a divided Congress, resorted to using the banking system as a means to implement policy through non-traditional channels. This effort, known as Operation Choke Point, was a continuation of their success in cutting off the offshore online poker industry from banking services. Initially, the crackdown was aimed at the payday lending industry, but it soon expanded to include gun sales and adult entertainment, and eventually up to 30 different industries.\par
The rationale behind Operation Choke Point was to target banks that facilitated fraud, as indicated by a high ratio of fraud and disputes. However, the operation soon evolved into a redlining of industries based on nothing more than the perceived risk of reputational harm. Financial institutions were investigated without any evidence of losses. Throughout the entire operation, there was no new legislation or written guidance issued. Banks were simply warned of increased regulatory scrutiny if they did not comply.\par
Major banks continue to deny services to industries such as firearms and fossil fuels, and they continue to assign higher risk ratings to industries that may face government criticism, even in the absence of any official guidance. This utilisation of the financial system as a means of driving change is seen by some as a legitimate, if not ideal, mechanism; as just one more type of market actor. Regardless of one's political perspective, it is important to consider the moral hazard of bypassing traditional political channels and using bureaucratic mechanisms as a means of affecting change in the free market. It is important to consider how the power of these tactics might be used in the future by opposing political groups. For example, supporters of Operation Choke Point who were in favour of increased financial pressure on the oil and gas industry may not feel the same if the same techniques were applied to organizations like Planned Parenthood. From this perspective, the tactics used by Operation Choke Point can be seen as undemocratic, regardless of who is deploying them. Bringing this back to our study of new financial tooling in crypto we can look to recent events:

\begin{itemize}
\item January: Some banks start to wind down activity in the crypto industry
\item January 21st: Binance announces its banking partner, Signature Bank, refuses to process Swift payments for less than \$100,000
\item January 27th: Federal Reserve denies Custodia Bank's application to access Federal Reserve System
\item January 27th: Federal Reserve denies Custodia Bank's application for a master account
\item January 27th: Federal Reserve releases statement discouraging banks from holding crypto assets or issuing stable coins
\item January 27th: National Economic Council issues policy statement discouraging banks from transacting with crypto assets or maintaining exposure to crypto depositors
\item February 2nd: DOJ announces investigation into Silvergate Bank over dealings with FTX and Alameda Research
\item February 6th: Binance announces suspension of USD bank transfers to and from offshore exchange
\item February 8th: Binance announces search for another banking partner
\item February 7th: Fed's policy statement enters Federal Register as a final rule
\item Two outstanding applications for National Trust Bank licenses from Anchorage and Paxos likely to be rejected by the OCC
\item Banking services becoming increasingly difficult for crypto firms, some startups will likely now not make the attempt
\end{itemize}
It seems that in the absence of democratic the SEC is attempting to use their tools to control and centralise the `ramps' into and out of digital assets, and the rules around holding them for investors. The SEC has proposed a new rule that would require registered investment advisors to use qualified custodians for all assets, including cryptocurrencies. The intention behind this proposal is to improve investor protection by mandating that custodians hold customer assets in segregated and identifiable accounts. However, critics argue that this proposal would limit the number of qualified cryptocustodians and deter investment advisors from advising their clients on crypto. The few banks with the necessary technical capabilities and regulatory approvals will have a monopoly on crypto custodial services, while exchanges without a banking license or trust bank will likely lose out. The proposal assumes that crypto assets are securities without going through a process to determine that. The outcome of the proposal will depend on the stringency of the SEC's qualified custodian registrations. The proposal is currently in a 60-day public comment period before the Commissioners hold another vote on whether to pass the rule.\par
Caitlyn Long explains that the proposed rule would not necessarily kill crypto custody, but would be a move against State Charter trust companies. She points out the big issue with the proposal, which is the requirement for custodians to indemnify for negligence, recklessness, or willful misconduct. This would apply to all asset classes, including commodities and crypto, which could kill the custody business broadly. The SEC proposal would apply the custody rule to all asset classes, including commodities and crypto, which is okay, but the SEC also wants custodians to indemnify the full asset value for losses in which the custodian played any role, even for physical assets like oil, cattle, and wheat. This would upset long-standing insurance terms and could cause huge pushback from the banking, Wall Street, commodities, and crypto industries. Sarah Brennan believes that the proposal represents continued governmental efforts at denial of service attacks on crypto, and that the SEC's approach only seeks to chill digital asset markets. She and the Republicans on the House Financial Services Committee are urging stakeholders to submit public comments on the proposed amendments to ensure the custody rule for investment advisors is modernized appropriately. It might be that the industry follows the prevailing winds \href{https://noelleacheson.substack.com/p/weekly-feb-25-2023}{and pivots to the East}. As usual, none of this particularly impacts our use case and thesis.
\section{Global monetary policy}
The term ``don't fight the Fed'' has been used in trading circles for many years. Owing to the pre-eminent role of the dollar in global markets actions of the political and central banking bodies which impact the dollar always have global reach. It is worth knowing that these decisions are usually contested, and worse, the power of the decision makers seems rooted in their narrative impact. It's a pretty terrible system given the impact on billions of lives. The Federal Reserve System, which is comprised of a Board of Governors, 12 regional banks, and an Open Market Committee, is a privately-owned central banking system in the United States. The member banks of each Federal Reserve Bank vote on the majority of the Reserve Bank's directors and the directors vote on members to serve on the Open Market Committee, which determines monetary policy. The president of the New York Federal Reserve Bank is traditionally given the vice chairmanship of the Open Market Committee and is a permanent committee member. This means that private banks are the key determinants in the composition of the Open Market Committee, which regulates the entire economy. The Federal Reserve is an independent agency and its monetary policy decisions do not have to be approved by the President or anyone else in the executive or legislative branches of government. The Fed's profits are returned to the Treasury each year, but the member banks' shares of the Fed earn them a 6\% dividend. The 2008 financial crisis and subsequent bailouts exposed the fundamental conflicts of interest at the heart of the Federal Reserve System, where the very banks that caused the crisis were the recipients of the trillions of dollars in bailout money. These conflicts of interest were baked into the Federal Reserve Act over 100 years ago and are a structural feature of the institution. The concentration of power within this group is staggering. 
\section{Opportunities in Africa, and Gridless}
In the course of researching this book we see most opportunity for change in Africa. As an example the company `Gridless' began by examining different energy sources in Africa and exploring opportunities for larger energy generation and grid-connected energy. However, they found that the real benefit of gridless energy was in providing energy to places that were not well connected and did not have a good grid. They contacted mini-grid providers all over East and Southern Africa to learn about their problems. A mini-grid is defined as a project that generates energy under 2 megawatts, often under 1 megawatt. They discovered that these providers had to overbuild for the community, resulting in stranded energy. The company found a way to utilize this stranded energy by placing Bitcoin miners on it and paying the mini-grid providers for it. They tested this method and found it to be successful. Additionally, they implemented a system to automate and remotely turn off the power during periods of high usage to make the grid more efficient and sustainable. This solution provided a win-win-win situation for the company, the mini-grid providers, and the communities they served.\par
The company utilizes Bitcoin miners to create space for other activities and to increase access to affordable energy for communities and small businesses. As energy usage increases in the community, the company decreases their usage of miners and moves them to other locations. This is outlined in their contracts with partners. The company is currently testing this method and has encountered some challenges, such as losing internet connection at one of their sites and poor rainfall affecting the amount of water flowing into turbines. They have found that building a lean operation with flexible and adaptable staff is crucial, as well as creating processes and systems to manage variables. The company also faces unique environmental factors such as lightning strikes, which require them to turn off their operations temporarily.\par
Gridless suggest that those who are critical of opportunities like this often come from a place of privilege and do not understand the consequences of their actions in places like Africa where access to electricity and other resources is limited. They argue that these critics, who are often from the West, have blinders on and cannot see the impact of their actions on a global scale. They suggest that more people need to travel and have diverse experiences in order to change their perspective on Bitcoin and its potential to support human flourishing in underprivileged areas. They also mention that gridless plans may become a case study for the positive impact of Bitcoin mining on economic opportunities, particularly in rural Africa.
\section{El Salvador as a case study}
El Salvador became the first country in the world to adopt Bitcoin as legal tender. El Salvador's adoption of Bitcoin was a historic moment in the world of Bitcoin and was met with a mix of excitement and scepticism. On June 9, 2021, the country's Legislative Assembly approved a bill introduced by President Nayib Bukele to make Bitcoin a legal tender alongside the US dollar, which has been used as the country's official currency since 2001.\par
President Bukele, who has been a vocal proponent of Bitcoin, stated that the adoption of Bitcoin was a way to promote financial inclusion and stability in the country, where more than 70\% of the population is unbanked or underbanked. In a tweet, he stated, ``Bitcoin will have the same value as the US dollar. We will support both. They will have the same power of purchase and will be accepted in the same way.''\par
The move was met with a lot of media attention and reaction, with some praising it as a bold and innovative step, while others raised concerns about the volatility of Bitcoin and their potential impact on the economy. President Nayib Bukele himself has faced criticism for his handling of political power and some of his actions have raised concerns about the potential for abuses of power. In 2021, President Bukele faced widespread criticism for his handling of the legislative process and his use of the military to secure the Legislative Assembly building during a political standoff with lawmakers. This led to allegations of intimidation and a violation of democratic norms, and raised concerns about his willingness to use force to achieve his political goals. Additionally, President Bukele has faced criticism for his use of social media to communicate with the public and his tendency to bypass traditional media outlets, which has raised concerns about the potential for censorship and the manipulation of information. With that said he seems much loved in the country, and the previously appalling safety statistics of the nation have radically improved.\par
In addition to the adoption of Bitcoin as legal tender, El Salvador has also proposed the issuance of a Bitcoin-backed bond to finance various public works projects and promote the use of Bitcoin. The bond would be denominated in Bitcoin and would allow investors to directly participate in the country's development while also supporting the growth and adoption of Bitcoin.\par
Another ambitious project that has been proposed by President Bukele and his administration is the creation of ``Bitcoin City'', a new city that would mine Bitcoin at the base of a dormant Volcano, and offer considerable tax benefits to holders. The city would serve as a hub for innovation and a showcase for the potential of Bitcoin, and would offer a wide range of services, including housing, healthcare, education, and entertainment.\par
There has been a significant increase in the adoption of Bitcoin in El Salvador, and apparently increased inward investment to the country. Many businesses, both small and large, have started accepting Bitcoin as a form of payment, and there has been a growing interest in Bitcoin among the general population. Additionally, the government has been actively promoting the use of Bitcoin through various initiatives. There have also been efforts to educate the public about Bitcoin and its potential benefits, including increased financial security and reduced transaction fees compared to traditional banking systems.\par
Overall, the adoption of Bitcoin in El Salvador has been positive, far outstripping the number of people in the country with traditional bank accounts, and has the potential to greatly impact the country's economy and financial sector. However, it is important to note that there are still challenges to overcome, such as regulatory and infrastructure limitations, as well as ongoing concerns about the volatility and stability of Bitcoin.\par 
Somewhat surprisingly the IMF have de-escalated their previously highly critical assessment of the move, toward a more \href{https://www.imf.org/en/News/Articles/2023/02/10/el-salvador-staff-concluding-statement-of-the-2023-article-iv-mission}{concerned and conciliatory tone}:
\textit{``Bitcoin’s risks should be addressed. While risks have not materialized due to the limited Bitcoin use so far—as suggested by survey and remittances data—its use could grow given its legal tender status and new legislative reforms to encourage the use of crypto assets, including tokenized bonds (Digital Assets Law). In this context, underlying risks to financial integrity and stability, fiscal sustainability, and consumer protection persist, and the recommendations of the 2021 Article IV remain valid. Greater transparency over the government's transactions in Bitcoin and the financial situation of the state-owned Bitcoin-wallet (Chivo) remains essential, especially to assess the underlying fiscal contingencies and counterparty risks.''}\par
In terms of economic impact, it is still too early to determine the full effects of the adoption of Bitcoin in El Salvador. However, it is expected to have a positive impact on financial inclusion and stability, as well as reducing the reliance on traditional banking systems. The use of Bitcoin has the potential to lower transaction fees and increase financial security, which could be particularly beneficial for those who do not have access to traditional banking services.\par
Overall, the adoption of Bitcoin in El Salvador marks a significant step forward in the mainstream acceptance and adoption of Bitcoin and has the potential to set a precedent for other countries to follow. However, it is important to monitor the situation and assess the long-term impacts on the economy and financial sector.