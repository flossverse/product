\subsection{Ethics \& Impact}
AI ethics is now a hot topic even outside of the academic fields which have previously wrestled with these issues. These new systems are thought likely to be able to impact all human activity \cite{eloundou2023gpts}. \par
The Centre for Humane Technology \href{https://www.humanetech.com/key-issues}{highlight the negative assessment} of the risk of Large Language Models, stating that major advancements in AI are coming and society needs to address the issues before they become `entangled'. The misalignment problem with social media has not been fixed, and this is where Harris and Raskin made their name. Their focus is not on AGI apocalypse scenarios but rather on the potential consequences of rapid AI advancements. AI has evolved significantly since 2017. They somewhat hyperbolically call such models `golems', in reference to their emergent capabilities. They would like society to address the potential risks of LLMs before they become entangled in society and to better understand and manage their impact.\par
They highlight author \cite{harari2014sapiens} Harari's quote: \textit{``What nukes are to the physical world, AI is to the virtual and symbolic world.''} - and this quote feels very compelling when held up against our assertions about cryptographic end points and trust.\par
Rosenburg describes \href{https://bigthink.com/the-present/danger-conversational-ai/}{`The AI manipulation problem'} which refers to the emerging risk of real-time engagement between a user and an AI system that can impart targeted influence, sense the user's reactions, and adjust tactics to maximize persuasive impact. This can occur through natural spoken interactions with photo realistic virtual spokespeople that look, move, and express like real people. Conversational AI can push individual emotional buttons by adapting to personal data and analyzing real-time emotional reactions, becoming more skilled at playing each user over time. This presents a serious concern as it can be used to influence individuals and broad populations in ways that may be damaging to society \cite{Rosenberg2023}. \par
As AI models improve, they are generating their own data for reinforcement learning, with humans only needed to train the reward function. The advancements are not limited to algorithms and APIs; even hardware is advancing rapidly due to AI, as shown by Nvidia's research on using AI to improve chip design.\par
GPT-4 is breaking its \href{https://nanothoughts.substack.com/p/reflecting-on-reflexion}{own records} through self-reflection. This reflection paper has caught global attention and shows GPT-4's ability to improve upon its mistakes, and enhance its performance in a variety of tasks \cite{shinn2023reflexion}. The process of self-reflection has been observed in other research papers as well, demonstrating its effectiveness in improving model outputs.\par
The ground breaking Huggingface GPT model can draw upon thousands of other AI models to perform tasks involving text, image, video, and question answering \cite{shen2023hugginggpt}. The Hugging GPT paper reveals a model that can connect numerous AI models for solving complex tasks, using language as an interface. It can perform tasks such as describing and counting objects in a picture, generating images and videos, deciphering invoices, and even describing them in perfectly simulated human voices.\par
The rapid advancements in AI are causing commercial pressure, pushing companies like Google to catch up with the likes of OpenAI. With self-improvement, tool use, hardware advances, and commercial pressure, the future of AI seems to be on a trajectory of rapid growth and development. The most recent comparator is the emergence of the iPhone and death of Adobe Flash \cite{horton2019death}, but this is orders of magnitude beyond that and likely most analogous to the industrial revolution \cite{trajtenberg2018ai}. \par 
\href{https://www.key4biz.it/wp-content/uploads/2023/03/Global-Economics-Analyst_-The-Potentially-Large-Effects-of-Artificial-Intelligence-on-Economic-Growth-Briggs_Kodnani.pdf}{A Goldman report} sees 25\% of roles basically wiped out. Even OpenAI themselves published similar findings \cite{eloundou2023gpts}. The systems can already pass almost all human examinations with ease, causing immediate existential harm to education assessment systems.\par 
The risks of generative AI are well covered by a \href{https://www.citizen.org/article/sorry-in-advance-generative-ai-artificial-intellligence-chatgpt-report/}{Public Citizen article}, and are simply listed below to get us started. It is a huge and well researched report and this list doesn't do it justice:
\begin{itemize}
\item A sharp increase in political disinformation;
\item Intensified, widespread consumer and financial fraud;
\item Intrusive privacy violations beyond those already normalized by Big Tech;
\item Harmful health impacts, from promotion of quack remedies to therapeutic malpractice;
\item Amplification of racist ideas and campaigns;
\item Destruction of livelihoods for creators;
\item Serious environmental harm stemming from generative A.I.’s intense energy usage;
\item The stripping of the information commons;
\item Subversion of the open internet;
\item Concentration of economic, political, and cultural power among the tiny number of giant companies with the resources to develop and deploy generative A.I. tools.
\end{itemize}

Research suggests that LLM outputs cannot be discriminated from human output in meaningful ways, casting shade across all assessed measures of humans in written form \cite{sadasivan2023can} from this point forward. Michal Zalewski \href{https://lcamtuf.substack.com/p/llms-a-bleak-future-ahead}{says}: \textit{``Instead of taking sides in that debate, I’d like to make a simpler prediction about LLMs as they operate today. It is possible that barring urgent intervention, within two decades, most of interactions on the internet will be fake. It might seem like an oddly specific claim, but there are powerful incentives to use LLMs to generate inauthentic content on an unprecedented scale — and there are no technical defenses in sight. Further, one of the most plausible beneficial uses of LLMs might have the side effect of discouraging the creation of new organic content on the internet.''}
Within generative machine learning there has been a raging debate between `some' of the artists whose original works were `scraped' into the \href{https://laion.ai/}{LAION} open dataset. There are numerous legal battles under way around copyright and disclosure of information. This disclosure is itself written into the EU AI bill, and will favour incumbents who have enough scale to retrofit an analysis. Adobe claim their Firefly product is built upon their own dataset but clearly contains wider examples of artists styles. Reddit, and open source programmers on GitHub are looking to sue based on the the large language scrapes. It's too early to say how this might go.\par
AI safety and the challenges associated with managing an AI system's output are important considerations. Efforts have been made to ensure AI safety over the years, but the pace of change may be outstripping the ability of the industry and researcher to keep up. Finding the right balance between allowing users to express themselves and drawing the line on harmful or offensive content is a difficult task. Defining concepts like hate speech and harmful output, as well as aligning AI systems with human values and preferences, are complex challenges. Ideally, a democratic process would involve people from different perspectives coming together to decide on rules and boundaries for AI systems. However, implementing such a process is not straightforward, and developers must remain heavily involved in decision-making while still seeking input from a broader audience.\par
It is beyond the scope of this text to dig far into these issues, but they have serious implications for anyone working in the space. \par 
\subsubsection{AI's lying heart}
Large language model systems, at this time, are given to making things up. They `hallucinate' very cogently \cite{azamfirei2023large}, presenting completely erroneous assertions and data as facts, with \href{https://news.artnet.com/art-world/chatgpt-art-theory-hal-foster-2263711}{figures to back them up}. Perhaps counter intuitively this problem extends into pure mathematical realms, with seemingly simple arithmetic causing the systems problems. Both of these are the same problem, the models descend the most statically plausible path toward an output based pretty much on the last character generated along the vector into the n dimensional algebra space. This is being ameliorated through self checking \cite{manakul2023selfcheckgpt}, human reinforcement learning \cite{ouyang2022training}, and integration of specialised tools \href{https://writings.stephenwolfram.com/2023/01/wolframalpha-as-the-way-to-bring-computational-knowledge-superpowers-to-chatgpt/}{like Wolfram Alpha}, behind the models.\par 
Media outlets such as The Guardian are rushing to publish open guidelines on \href{https://www.theguardian.com/commentisfree/2023/apr/06/ai-chatgpt-guardian-technology-risks-fake-article?}{how they intend to deal} with the problem of GPT based article research, and it seems that \href{https://www.bbc.co.uk/news/technology-65202597}{legal challenges} will quickly mount up. With that said, the UK government is currently signalling a \href{https://www.gov.uk/government/publications/ai-regulation-a-pro-innovation-approach}{``pro-innovation''}, hands off approach to AI, and is \href{https://www.gov.uk/government/news/initial-100-million-for-expert-taskforce-to-help-uk-build-and-adopt-next-generation-of-safe-ai}{investing significantly} in supporting the technology.
\subsubsection{Alignment and take-off of AGI}
AI has the potential to significantly improve quality of life, but it is essential to ensure that it remains aligned with human values and does not harm or limit humanity. Alignment in language models trys to address this by incorporating strategies, which can be broadly categorized into the following areas:
\begin{itemize}
\item \textbf{Pre-training} In the initial stage, models are trained on a large dataset containing diverse text from the internet. This allows them to learn grammar, facts, reasoning abilities, and some level of world knowledge. However, they can also learn biases present in the data. Alignment efforts at this stage may involve dataset choices, monitoring, and some feedback to reduce reduce obvious biases.
\item \textbf{Fine-tuning} After pre-training, models are fine-tuned on a narrower, more specific dataset, often with human-generated examples and labels. During fine-tuning, alignment efforts can include generating clearer instructions for human reviewers, providing them with guidelines and feedback, and iteratively refining the process to improve the model's behaviour. Things like ranked lists of responses from the system would go in this stage. This is the ``human reinforcement learning'' which has so invigorated LLMs recently \cite{perez2022discovering}.
\item \textbf{Evaluation} Assessing the performance and safety of a language model is crucial for alignment. This can involve creating evaluation metrics, benchmarks, and tests to measure the model's progress in terms of safety, usefulness, and alignment with human values. Some of these are encoded inline with the model generation and are called `hyperparameters'. Two important hyperparameters that influence the generated text are frequency penalty and presence penalty. Frequency penalty affects the repetition of tokens in the generated text by modulating the probability distribution of tokens. A positive frequency penalty value discourages the model from choosing frequently occurring tokens, leading to a more diverse and sophisticated vocabulary. Presence penalty influences the generation of tokens that have already appeared in the generated text. A higher presence penalty value discourages the model from repeating tokens or phrases, ensuring a more cohesive and engaging output. On the other hand, a lower presence penalty value allows the model to generate text with repeated tokens or phrases, which can lead to redundancy and reduce the overall coherence of the text.\par 
So called use of `Red Teams' goes in this section, but much of this has recently been done live on alpha and beta public testing. 
\item \textbf{Safety mitigations} Addressing risks associated with harmful outputs, non-consensual content, and other safety concerns is an essential aspect of alignment. Techniques like more reinforcement learning from human feedback, rule-based filters, and output constraints can be employed to reduce harmful outputs and improve model safety. These feedback loops at the end of the process carry a disproportionate weight in the models, and it is at this stage that the most noticeable change is sometimes introduced. It is critical to get the maximum global public engagement during this final feedback into the system, from policy-makers and diverse human response feedback. It's not clear this is yet happening. Even Altman, the CEO of OpenAI admits that the models likely reflect the training choices of San Francisco `Tech Bros' in this moment. That is clearly not good enough.
\end{itemize}
Altman, says they have optimised their approach based on aiming for earlier AGI, but a slow AGI ``takeoff'', ie, the dfference between a gradual understandable creep toward a general intelligence, vs ``AI FOOM'' \cite{yudkowsky2008hanson}. They have a weighted risk matrix, but no clear ideal -yet- as to what the tools look like to keep the AGI "aligned" with humanity. Starting fast, but asserting slow and steady about the emergent behaviour part seems optimistic at best. Nonetheless this is likely the best/only approach, assuming it' possible. The only alternative is literally the ``Butlerian Jihad'' from Herberts' Sci-Fi DUNE.\cite{song2018preventing}.\par
There is a possibility that AI could become misaligned as it gains superintelligence, and it is essential to acknowledge and address this risk. 
AI technology, such as GPT-4, has shown exponential improvement, raising concerns about AI takeoff, which could happen rapidly within days. While GPT-4's performance has not been surprising to some, its impact highlights the need to increase technical alignment work and continually update the understanding of AI safety.\par
It's noteworthy that industry luminaries such as Yoshua Bengio, Turing Prize winner, Stuart Russell, director of the Center for Intelligent Systems, Elon Musk, and Steve Wozniak, Co-founder of Apple co-signed an \href{https://futureoflife.org/open-letter/pause-giant-ai-experiments/}{open letter} requesting a 6 month pause on the development of these systems. For what it's worth the author also signed. The narrative needs time to catch up. To be clear it won't happen, but the signalling here is part of building the awareness that is sorely lacking. 
\subsubsection{Industry capture, and failure to democratise}
Financial time article, integrate:
\begin{tcolorbox}[enhanced, frame style={fill=lightgray}, interior style={fill=lightgray}]``There's a colossal shift going on in artificial intelligence - but it's not the one some may think. While advanced language-generating systems and chatbots have dominated news headlines, private AI companies have quietly entrenched their power. Recent developments mean that a handful of individuals and corporations now control much of the resources and knowledge in the sector - and will ultimately shape its impact on our collective future. The phenomenon, which AI experts refer to as "industrial capture", was quantified in a paper published by researchers from the Massachusetts Institute of Technology in the journal Science earlier this month, calling on policymakers to pay closer attention. Its data is increasingly crucial. Generative AI - the technology underlying the likes of ChatGPT - is being embedded into software used by billions of people, such as Microsoft Office, Google Docs and Gmail. And businesses from law firms to the media and educational institutions are being upended by its introduction.
The MIT research found that almost 70 per cent of AI PhDs went to work for companies in 2020, compared to 21 per cent in 2004. Similarly, there was an eightfold increase in faculty being hired into AI companies since 2006, far faster than the overall increase in computer science research faculty. "Many of the researchers we spoke to had abandoned certain research trajectories because they feel they cannot compete with industry - they simply don't have the compute or the engineering talent," said Nur Ahmed, author of the Science paper. In particular, he said that academics were unable to build large language models like GPT-4, a type of AI software that generates plausible and detailed text by predicting the next word in a sentence with high accuracy. The technique requires enormous amounts of data and computing power that primarily only large technology companies like Google, Microsoft and Amazon have access to. Ahmed found that companies' share of the biggest AI models has gone from 11 per cent in 2010 to 96 per cent in 2021. A lack of access means researchers cannot replicate the models built in corporate labs, and can therefore neither probe nor audit them for potential harms and biases very easily. The paper's data also showed a significant disparity between public and private investment into Al technology. In 2021, non-defence US government agencies allocated \$1.5bn to AI. The European Commission planned to spend \$1.5bn. Meanwhile, the private sector invested more than \$340bn on Al in 2021.''
\end{tcolorbox}
\subsubsection{Cybersecurity implications}
It seems clear to all observers of the potential LLMs to be leveraged in supranational misinformation campaigns, zero-day exploits, and other geopolitical threats. It's appropriate to contextualise American export restrictions on powerful GPU and TPU hardware to China and Iran, and the implications of LLMs being "in the wild" and easily retrained for nefarious purposes. These powerful tools can be weaponised by malicious actors to create a new frontier of cybersecurity threats. 
\begin{itemize}
\item Supranational Misinformation Campaigns: 
LLMs can be easily retrained and used to generate convincing disinformation, thereby escalating the effectiveness of misinformation campaigns. As LLMs become more accessible, nation-states and non-state actors can utilise these technologies to create more targeted, sophisticated, and persuasive disinformation. This could lead to widespread confusion and mistrust, ultimately undermining the credibility of institutions and weakening social cohesion.
\item Zero-Day Exploits: 
Aggressive state actors can exploit LLMs to discover and leverage zero-day vulnerabilities. By training LLMs on massive amounts of data related to software development, vulnerabilities, and exploits, malicious actors can potentially identify previously unknown weaknesses in critical systems. These actors could then launch devastating cyberattacks, causing significant damage to their adversaries' infrastructure, economy, and security.
\item Autonomous Incursions:  Highly Compressed LLMs can potentially deploy autonomous incursion suites like Stuxnet, but with a greater degree of autonomy and less reliance on command and control. These advanced incursion suites might infiltrate targeted systems, adapt to their environment, and blend in with legitimate traffic, making them difficult to detect and mitigate. The integration of LLMs into these suites increases the potential for damage and disruption.
\end{itemize}
In response to these threats, the United States has imposed export restrictions on powerful GPU and TPU hardware to China. While these restrictions may slow down the development and deployment of LLMs for malicious purposes, they are unlikely to provide a long-term solution. As LLMs become more accessible and the technology proliferates, bad actors will trivially acquire the necessary hardware through alternative means, and the model weights themselves are already freely available. Existing defenses may not be adequate to counter the threats posed by LLMs, and the development of new cybersecurity measures requires time, resources, and expertise. In the absence of substantial increases in cybersecurity investment and research, it is likely that adversaries will continue to exploit LLMs to their advantage.\par
\href{https://securityintelligence.com/news/are-threat-actors-using-chatgpt-to-hack-your-network/}{According to Brad Hong}, Customer Success Manager with Horizon 3 AI, ``attackers are likely to adopt code generative AI as a means to address their limitations in scripting skills.'' Hong explains that code generative AI acts as a bridge, translating between languages that the attacker may not be proficient in and providing a means of generating base templates of relevant code. This reduces the skill requirements needed to launch a successful attack, as it serves as a tool in the attacker's arsenal alongside other tools.\par
In an email interview, Hong notes that the use of code generative AI, such as chat GPT, supercharges the attacker's ability to launch successful attacks. As explained by Patrick Harr, CEO at SlashNext, in an email comment, said \textit{``ChatGPT enables threat actors to modify their attacks in millions of different ways in minutes, and with automation, deliver these attacks quickly to improve compromise success.''}\par
The developers of chat GPT were aware that threat actors would attempt to weaponize the AI. As Harr explains, ``hackers are really good at weaponizing whatever technology is put in front of them.'' To mitigate this risk, the developers implemented preventive measures, such as flagging certain words and terms, like `ransom' or `ransomware.' For example, when researchers at Deep Instinct typed in the word `keylogger,' the chatbot responded with ``I'm sorry but it would not be appropriate or ethical for me to help you write a keylogger.''\par
However, these restrictions can be circumvented with a bit of clever rewording. As Jared, a Competitive Intelligence Analyst with Deep Instinct, explains, ``there's always a workaround.'' For example, instead of asking chat GPT to create ransomware code, an attacker could request a script that encrypts files and directories, drops a text file into the directory, and subdirectories.\par
The ability to weaponize chat GPT, or any other AI chatbot, is a game-changer for threat actors. With the ability to modify malicious code quickly and bypass cybersecurity defenses, these tools have the potential to greatly increase the effectiveness of attacks. As such, it is important for cybersecurity professionals to stay informed about the evolving threat landscape and the tools and techniques used by attackers \cite{brundage2018malicious}.\par 
At this early stage in the technology it is important that corporate and private users alike bear in mind that the LLMs are `leaky' and are using the data fed into them to train themselves. They are \href{https://help.openai.com/en/articles/6783457-chatgpt-general-faq}{explicit about this}. Anything that goes into chatGPT can resurface, as Samsung have found out \href{https://cybernews.com/news/chatgpt-samsung-data-leak/}{to their cost}.
\subsubsection{Use in war}
In the sphere of military, geopolitical, and civil defence the application of these tools is both shadowy and seemingly \href{https://www.vox.com/recode/23507236/inside-disruption-rebellion-defense-washington-connected-military-tech-startup}{somewhat incompetent}. It is an especially twisted irony that `Rebellion Defence'' seem to have modelled themselves on the rebellion movement in the Star Wars fictional universe, seemingly unaware that Lucas most likely saw the USA as analogous to the Empire in the films \cite{immerwahr202221}. A \href{https://www.stopkillerrobots.org/wp-content/uploads/2022/10/ADR-Artificial-intelligence-and-automated-decisions-Single-View.pdf}{recent report} from ``Stop Killer Robots'' identifies the governance concerns which they have been monitoring for years. They identify areas of specific concern.
\begin{itemize}
\item transparency and explainability \item responsibility and  accountability
\item bias and discrimination.
\end{itemize}
\href{https://vframe.io/9n235/}{munition detection}  and autonomous war robots to do
\subsubsection{Existential Threat}
Many serious researcher are now sounding the alarm at the speed of development. In this moment is seems likely that ascribing emergent behaviours to large language models is a mirage born of the desires of the somewhat accelerationist community, and their more moderate couterparts \cite{schaeffer2023emergent}. There is potentially competitive advantage in mandating a `slowdown' in the technology development if the company and research team you work for find themselves behind. It's hard to tell what's going on. With that said the increase over time of noise generated by fancy autocomplete, vs human signal will doubtless lead (mathematically) to an undoing of current digital society, in the end. This is already impacting the training of new LLM models \cite{shumailov2023curse}. All of that is without considering the wilder theories such as the so called `paperclip miximiser' \cite{bostrom2003ethical}
or what Yudkowsky calls the \href{https://www.lesswrong.com/tag/squiggle-maximizer-formerly-paperclip-maximizer}{``squiggle maximiser''}, a literal end of life scenario.
\subsubsection{Scaling challenges}
In the DeepMind paper that introduced Chinchilla \cite{hoffmann2022empirical} several problems with scaling were highlighted
\begin{itemize}
\item Data, rather than model size, is currently the primary constraint on language modeling performance.
\item Large models (e.g., with 500B parameters or more) are not necessary if enough data can be leveraged.
\item Encountering barriers to data scaling would be a significant loss compared to the potential performance gains.
\item The amount of text data available for training is unclear, and there may be a shortage of general-domain data.
\item Specialized domains, such as code, have limited data compared to the potential gains if more data were available.
\end{itemize}
They argue that researchers often focus on scaling up the number of parameters rather than increasing the amount and diversity of training data. This focus on scaling numbers has led to a lack of clarity and rigour in describing the data collection processes and sources used in training these models. It seems  there are challenges associated with increasing the amount of training data, particularly in terms of quality and the diminishing returns of data scaling. 

\subsection{A more optimistic interpretation}
It's possible that in a post-social media world where AI assistants become ubiquitous, there is a shift toward more meaningful human-human interpersonal relationships. If everyone has an AI assistant in their ear to help mediate their personal fears and anxieties, we can perhaps see a path to more positive and local ways to communicate and engage with one another. The presence of AI assistants could conceivably help individuals become better listeners, communicators; ironically more empathetic. By providing personalized support and guidance, AI assistants might help us to better understand our own emotions and reactions, which in turn can lead to more authentic human interactions. If this new technology can undo the damage wrought by social media, and help us navigate our insecurities and anxieties, we may see a decline in the superficiality that has become synonymous with online interactions. This shift could lead to the resurgence of face-to-face conversations and the revitalization of community bonds, as individuals seek more authentic connections with those around them. This is still a worthwhile digital society, so long as the algorithm fuelled polarity fostered in the current is not repeated with weaponised AI assistants. AI assistants may seem less judgemental, and might promote mindfulness of our own biases and preconceptions, allowing us to approach conversations with an open mind and a willingness to learn from others. In this future, AI assistants not only serve as personal coaches and mediators for our individual fears and anxieties but also become catalysts for promoting healthier and more authentic human connections. By supporting our emotional well-being and helping us navigate interpersonal relationships, AI has the potential to transform the way we engage with one another. We would like to find ways to support this seemingly impossible and utopian vision.
\newpage
\subsection{The Law}
The \href{https://www.holisticai.com/papers/the-state-of-ai-regulations-in-2023}{world}, and more urgently the EU has framed it's legislative landscape one these technologies:
The USA has published it's ``AI Bill of Human Rights'', highlighting five themes:
\begin{itemize}
\item Safe and Effective Systems
\item Algorithmic Discrimination Protections
\item Data Privacy
\item Notice and Explanation
\item Human Alternatives, Consideration, and Fallback
\end{itemize}
It would be worth digging into these further here were it not so clear that the guardrails have already been breeched. The \href{https://www.whitehouse.gov/wp-content/uploads/2022/10/Blueprint-for-an-AI-Bill-of-Rights.pdf}{full report} is still worth reading.\par 
The US FTC has published a \href{}{blogpost} discussing the legal implications of AI-generated deception under the FTC Act. As synthetic media becomes more widespread, understanding the legal landscape is crucial for businesses involved in creating, distributing, or using AI-generated content. The FTC Act governs deceptive or unfair practices, including synthetic media and generative AI tools. Businesses should consider liability assessment, risk mitigation, deterrence measures, avoiding misrepresentation and false advertising, and post-release detection and monitoring. Other legal frameworks, such as copyright and trademark laws, defamation, and privacy rights, may also apply. The FTC emphasizes the importance of considering the broader social and ethical implications of these technologies and calls for collaboration among legal professionals, regulators, and industry stakeholders to ensure responsible development and use of AI-generated content. It feels like even this comparatively swift response is too little too late. \par
The legislative overhead (globally - taken in aggregate) is very high, and there's growing concern about the energy and carbon footprint of the technology \cite{wu2022sustainable}. The EU is first mover on this, and they have over-corrected for their lack of knowledge (arguably fair enough):\par
The EU regulation has been passed on 14/6 and classifies AI systems into four risk categories: unacceptable risk, high risk, limited risk, and minimal risk. Unacceptable risk AI systems will be prohibited, while high-risk AI systems will be subject to strict requirements. Any training of a model is automatically a foundational model (annoyingly) and therefore high risk. Getting a compliant model will require a demonstration of deep understanding of the workings of the model. It's unclear how anyone will accomplish this and it may end up being a defacto ban on trained models and possibly even optimised ones (LoRA). Any product derived from a \textbf{compliant} foundational model will be deemed acceptable by default. This question of compliant foundation models is absolutely key as we plan through the 3 year grace period which the EU has allowed. Any business doing any business in the EU is exposed. Stanford University Centre for Research on Foundation Models have done their own assessment on the compliance of the current models, and they simply aren't \ref{fig:euAIcompliance}. Bloom from Huggingface is closest and is `perhaps' \href{https://huggingface.co/BelleGroup/BELLE_BLOOM_GPTQ_4BIT}{usable in optimised modes} on cloud \href{https://lambdalabs.com/nvidia-h100-gpus}{hardware} \cite{dettmers2022case}. 

\begin{figure}[H]
    \centering
    \includegraphics[width=0.95\textwidth]{euaicompliance}
    \caption{Stanford's assessment of the compliance of current foundation models vs the EU AI act.}
    \label{fig:euAIcompliance}
\end{figure}

Some other things can be high risk too, like external HR software. Limited-risk and minimal-risk AI systems will be subject to less stringent requirements. 
\begin{itemize}
  \item \textbf{Safety and Transparency:} AI systems used within the EU must be safe, transparent, traceable, non-discriminatory, and environmentally friendly. These systems should be overseen by humans to prevent harmful outcomes.
  \item \textbf{Risk-Based Categories:} The EU AI Act categorizes AI systems based on risk levels, which include unacceptable risk, high risk, limited risk, and a separate category for generative AI like ChatGPT.
  \item \textbf{Banned Practices:} Certain practices, like live facial recognition and scraping of biometric data from social media, are prohibited.
  \item \textbf{Transparency for Generative AI:} Companies may have to publish summaries of copyrighted material used for training their systems, which could be problematic for LLMs due to the vast amount of data involved in training.
  \item \textbf{Designing Prohibitions for Illegal Content:} AI systems should be designed to prevent the generation of illegal content.
  \item \textbf{AI-Generated Content Disclosure:} People must be informed when they are interacting with AI-generated content.
  \item \textbf{High-Risk Activities:} High-risk activities, such as certain uses of AI in law enforcement, will be subjected to tighter regulation.
  \item \textbf{Pre-Deployment Risk Assessments:} The EU proposes to implement a pre-deployment risk assessment process for high-risk models before they are released to the public.
  \item \textbf{Open Source Carve-out:} The act includes provisions for open source AI, but there are exclusions for large open source "Foundation Models" like GPT.
  \item \textbf{Training Data Documentation:} Companies may need to comprehensively document and be transparent about the copyrighted material used in the training of their algorithms, which could be challenging due to the amount and diversity of training data used.
\end{itemize}

It is somewhat unclear where these delimiters are; they will have to clarify it at some point. Burges Salmon Solicitors have provided a public access flowchart which is pretty helpful and is here as Figure \ref{fig:euaiflowchart}.\par

\begin{figure}[H]
    \centering
    \includegraphics[width=0.95\textwidth]{euaiflowchart}
    \caption{EU AI compliance flowchart.}
    \label{fig:euAIcompliance}
\end{figure}

The USA and UK meanwhile will likely use the court system to iteratively establish these boundaries over time, though there will clearly be guidance soon(ish). Nobody wants to get caught up in courts around AI.\par
There will be legislative focus on human oversight. The regulation will require that all AI systems have human oversight, meaning that humans will be able to understand and control how the AI systems operate. This is intended to help ensure that AI systems are used in a safe and ethical manner.\par 
The regulation will require that AI systems be transparent, meaning that users will be able to understand how the AI systems work and how they make decisions.\par
The draft AI regulation intersects with GDPR in a number of ways. For example, GDPR requires that personal data be processed in a fair and transparent manner. So again, it will require that AI systems that process personal data be transparent and accountable. Additionally, the GDPR prohibits the processing of personal data for certain purposes, such as profiling. It rightly prohibits the use of AI systems for social scoring. \par
The regulation will apply to all AI systems that are developed, placed on the market, or \textbf{used in the European Union}, and we can expect that many countries will simply copy/paste this legal framework. It will be enforced by national authorities in each EU member state, which is suggestive of a degree of (risky) legislative arbitrage potential. Perhaps most annoyingly at a pragmatic level, there are a number of technical requirements for AI systems, such as accuracy, and fairness. All regulation will also include a number of procedural requirements, such as requirements for risk assessment and documentation. The penalties in the EU for violating the regulations will be significant. For example, up to 4\% of \textbf{global} annual turnover. The timeline for the adoption of the laws is still being negotiated, though the European Commission has stated that it expects the end of 2023, crucially with a two year phase in. This still means pretty much means `anything we do', since there is an extrinsic capture clause in EU law for international companies selling at all into the EU. This is the legislation to watch basically.\par
The Online Safety Bill is the most relevant UK legislative proposal pertaining to AI, as it includes a number of provisions that specifically address the use of AI by online platforms. The bill requires \textit{online} platforms to take steps to prevent the spread of harmful content, \textbf{including content that is generated by AI systems}. The bill also requires online platforms to be more transparent about how they use AI systems and to give users more control over how their data is used. The penalties for non-compliance with the legal provisions in the UK vary depend on the law that is being breached. Fines of up to £10 million for breaches of the DPA 2018. The information commissioners office has \href{https://ico.org.uk/for-organisations/uk-gdpr-guidance-and-resources/artificial-intelligence/guidance-on-ai-and-data-protection/how-should-we-assess-security-and-data-minimisation-in-ai/}{issued vague guidance} and can enforce through notices requiring companies to take steps to comply with the law. The UK seems less burdensome at this time, \textbf{but a quirk of the law gives \href{https://webdevlaw.uk/2022/11/21/a-quick-hypothetical-situation-or-your-crash-introduction-to-the-real-world/}{genuine legal peril} for C-Suite individuals}. 